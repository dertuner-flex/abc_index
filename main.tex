\documentclass{article}

\usepackage{indentfirst}
\usepackage{amsmath,amsthm,amssymb}
\usepackage[utf8]{inputenc}
\usepackage[T2A]{fontenc}
\usepackage[russian, english]{babel}
\usepackage[colorlinks,allcolors=red]{hyperref}

\usepackage{graphicx}
\graphicspath{ {./images/} }

\title{The atom-bound connectivity (ABC) index and application}
\date{}

\begin{document}

\maketitle
\newpage 

\section{Введение}

Так как на данный момент отсутствует устоявшийся перевод для понятия atom-bound connectivity index (иногда можно встретить такой перевод как индекс атомной связности), будем просто называться его $ABC$ индекс. Этот индекс обязан своему появлению химическому смыслу который стоит за ним, но в данный момент мы отстранимся от этого и будем просто будем рассматривать этот индекс как некий инвариант графа (ниже покажем, что это на самом деле инвариант графа, то есть он не зависит от способа обозначения вершин или графического представления графа). На самом деле ABC индекс относится к чуть более широкому классу чем инварианты графа - топологические индексы графа, которые как можно понять могут иметь на изоморфных графах различные значения, но в дальнейшем мы не будем затрагивать это понятие.

Введем определение $ABC$ индекса. Пусть $G = <V, E>$ - неориентированный граф, где $V$ - множество вершин и $E$ - множество ребр. Тогда $ABC$ индекс графа $G$ имеет следующий вид:
$$ ABC(G) = \sum_{E} \sqrt{\frac{d_{u} + d_{v} + 2}{d_{u}d_{v}}} $$
Где сумма понимается в смысле по всем ребрам $(u, v) \in E$, $d_u$ и $d_v$ степени вершин $u$ и $v$ соответственно.

Далее отметим, что любой индекс, который имеет вид:
$$ ArbitraryIndex(G) = \sum_{E} f(d_{u}, d_{v}) $$

Имеет одинакое значение для любых двух изоморфных графов (то есть является инвариантом), это следует из того, что изоморфизм графов подразумевает под собой существование биективного отображения, которое сохраняет смежность вершин в графе, то есть каждая вершина в графе $G_1$ будет иметь столько же соседей сколько и сопоставленная ей вершина при изоморзиме в графе $G_2$ (при изоформизме степени вершни сохраняют), и далее можно утверждать, что отсортированный в неком порядке вектор из степеней вершин графа является константой в изоморфных между собой графах, и аналогично если мы для всех ребер графа выпишем список из пар: $\{(d_u, d_v), \; (u, v) \in E \}$, и отсортировав его по первому элементу (если они не равны, иначе по второму) получим объект который является инвариантом графа, и теперь можно утверждать, что рассматриваемое нами семейство индексов принимает одинаковые значения на изоморфных графах, то есть является инвариантом.

Конечно, чуда не произошло и наш индекс не является полным инвариантом графа. Напомним, что инвариант представляет собой попытку описания сложной сложной структуры графа неким более простым представлением, например числом. И в идеале нам хотелось бы что бы изофорфные между собой графы имели одинакой индекс, а не изофорфные обязательно различный. Но проблема состоит в том, что для произвольных графов не было найдено полиномиального алгоритма проверки на изоморфность (но также не доказано что его не существует), а следовательно существования на данный момент идеального в нашем понимания идекса обязывало его иметь экспоненциальную сложность вычисления. Но как мы видим из определения вычисление ABC индекса имеет асимптотическую сложность $O(|E|)$, и следственно можно утверждать, что существуют неизоморфные между собой графы которые имеет одинаковый по величине индекс. Ниже приводим пример.

\begin{figure}[h]
\includegraphics[scale=0.3]{image1}
\centering
\caption{Граф G1}
\end{figure}

\begin{figure}[h]
\includegraphics[scale=0.3]{image2}
\centering
\caption{Граф G2}
\end{figure}

Как видим выше, представленные графы  G1 и G2 не являются изоморфными, но при этом оба имеют ABC индекс равный $2 \sqrt{\frac{5}{3}} + 4 \sqrt{\frac{1}{2}} + \frac{1}{4} + \sqrt{6}$. 

Тут возникает вопрос, а каким образом образом нам искать графы, которые является неизоморфными, но при этом имеет одинаковый $ABC$ индекс ? К сожалению, нам неизвестный эффективный (понимается в смысле полиномиальный) алгоритм построения таких графов, как минимум причина этому, что мы утыкаемся в проблему изоморфизма, поэтому тут \hyperlink{first_bibitem}{[1]} можно найти реализацию которая ищет такие пары за экспоненциальное время. Алгоритм, который используется в нашей программе выглядит следующим образом:

1.) Фиксируем число вершин искомого графа, обозначим его через k.

2.) Генерируем все возможные простые неориентированные графы на k вершинах, коих $2 ^ {C_n^2}$.  

3.) Далее перебирая все возможные пары (коих $C_{2 ^ {C_n^2}}^{2}$ штук), каждый раз имеем зафиксированную пару графов, сравниваем $ABC$ индексы, и в случае равенства индексов проверяем графы на изоморфность.

Таким образом, мы получаем довольно трудозатратный по вычислительным мощностям алгоритм, который разумно применять для k не более 10, но тем менее легко видеть, что мы способно обнаружить все пары неизоморных графов с одинаковым ACB индексом, что доказывает корректность алгоритма.


\section{История}
Для начала опишем общую мотивацию возникновения ABC индекса. В некий момент возникает идея использовать предикативные модели в задачах аналитической химии, а следовательно возникает необходимость в построении векторного описания молекулы. Так же стоит вопрос какую информацию об молекуле нам необходимо использовать для построения достаточно "хорошего" маломерного описания (а нам желательно иметь маломерное векторное представление нашей молекулы, иначе работа с ним может вызвать проблемы (большие вычислительные затраты, шумовые компоненты, погрешности и так далее)). Достаточно ли использовать только структуру молекулы чтобы получить некие информативные представления ? Оказывается - да, можно, и более того, мы в этой работе не будем рассматривать способы добавить дополнительную информацию о молекуле исходя из ее химических свойст, что бы получить более качественные представления, а будем интересоваться только структурной состовляющей. 

ABC индекс позволяет получить некое одномерное представление молекулы, использую только ее топологические свойства (рассматриваем атомы как вершины и связи между атомами как ребра). И в последствии оказывается, что этого одномерного представления нам достаточно для получение некоторых хороших практических результатов. Но при этом, мы получаем не просто черный ящик, который способен по молекуле выдать нам ее одномерное векторное представление (или просто число), а граф, который соответствует структуре молекулы, и далее уже проводя некие манипуляции с ним, мы получаем нужное нам числовое описание молекулы. То есть, граф и определенный на нем ABC индекс, дают возможность использовать инструменты для исследования графов, что бы получать некоторые результаты для свойст ABC индекса.

Более формально, семейство таких моделей, которые по структурам химических соединений позволяют предсказывать их разнообразные свойства называется QSAR \hyperlink{qsar_link}{[2]} (Quantitative structure-activity relationship), другими словами это общее название регрессионных моделей и классификаторов, которые работают с векторными представления структуры молекулы.  

Контринтуитивным представляется тот факт, что мы пытаемся описать молекулу (которая сама по себе представляет довольно сложный объект) всего одним числом, и при этом надеемся, что это представление окажется информативным в некотором полезном практическом смысле. Ведь более логично выглядит, например, описать структуру молекулы (опять же, мы рассматриваем только топологическую состовляющую молекулы, не беря во внимание ее химические свойства) графом и далее применить некоторые графовые эмбеддинги (например, анонимные случайные блуждания \hyperlink{random_walk}{[4]}) и следующим шагом обучать модели машинного обучения уже на полученных таким образом векторных представлениях. Или, например, использовать более хитрые эмбеддинги (но еще не значит, что "хитрые" есть более "хорошие"), которые ориентруются на то, что граф представляет собой некую молекулу, такие как в этой работе \hyperlink{molecular_embedding}{[8]}. Да, в общем случае такой подход выглядит более целесообразным, так как мы стараемся извлечь максимальное количество информации из графов, и, например, в задачах классификации (например, таких как, бинарная классификация того, является ли молекула токсической или нет) будет работь явно лучше, чем просто описать все графы одним числом (в нашей случае ABC индексом). Но с другой стороны, пространство признаков (каждая компонента вектора эмбеддинга - признак) приобретает более сложную форму и соответственно искомая зависимость уже может быть достаточно нетривиальной, что в свою очередь накладывается ограничение на используемые методы машинного обучение (так, например, линейных моделей может быть уже не достаточно, потому что линейная аппроксимая совсем будет далека действительно), помимо этого мы теряем интерпретацию результатов и  лишаемся возможности анализировать граф методами теории графов, что бы получать различные свойства нашего представления. 

Подводя итог вышесказанного, скажем, что более сложные графовые эмбеддинги, хоть являются более информативными (в том смысле, что позволяет нам сохранить в некотором смысле максимальное количество информации о структуре графа) и показывают в среднем более хорошие результаты на широком спектре задач, но могут проигрывать в узконаправленных задачах. Таким именно и является наш случай с  ABC индексом, который имеет сильную корреляцию с некоторым химическим свойством молекулы, то есть мы линейную зависимость от одного параметра (ABC индекса), что в свою очередь позволяет нам делать качественные прогнозы о свойствах (точнее об одном конректном свойстве) молекулы по ее структуре.


Впервые ABC индекс упоминается в статье \hyperlink{fisrt_mention}{[2]}, а именно Estrada et al. предложил новый (на момент 1998 года) топологический индекс, который достаточно хорошо коррелирует с тепловым эффектом химической реации образования алканов. Как мы ранее упоминали, интерес к ABC индексу продиктовам именно его химическим подтексом, то есть рассматривая молекулу как граф и изучаю его свойста, мы получаем полезную практическую информацию.


В данной работе мы не будем касаться химического аспекта ABC индекса, нам будет интересовать идеологическая состовляющая. Например
Далее отметим, что графовое представлнеие молекулы - это просто некий иной способ сохранить информацию об объекте (в нашем случае молекуле) и далее применив инструменты для теории гравоф получать некоторые результата.


\newpage

\section{Математические концепты в химии}
Так как мы уже ранее коснулись химического подтекса ABC индекса, то полностью отречься от него уже не получится (иначе, например, не дать внятного ответа, почему рассматривать графы со степенью вершины не более 4, является не искусственным и бесполезным ограничением, а есть просто отражением физической реальности). С другой стороны необходимостью является (чтобы обрести понимание) представить в более ярко выраженном виде способы отображение из мира физических моделей в мир математики, тем самым получив способность применять широкий спектр инструментов для анализа тех математических структур, в виде которых нам удалось получить представление, и далее обратно проецировать результаты в мир химии. Конечно, эта глава не накладывает на нас никаких дополнительных требований в восприятии, и мы все так же можем воспринимать ABC индекс как некую функцию от графа, а граф просто как граф (возможно с некоторыми структурными ограничениями), не задумываясь о природе его происхождения.

Одним из естественных способов описать молекулу, есть взгляд на нее как на некий трехмерный объект \hyperlink{molecular_geometry}{[5]} (который имеет определенное пространственное расположение, следовательно помимо локального изучения самой молекулы, мы можем изучать взаимодействие молекул в совокупности), где основными состовляющими являются атомы (atoms) и связи между ними (bonds). Такой взгляд берет в себе начало со времен Якоба Хендрика Вант-Гоффа \hyperlink{first_people_stereochemistry}{[6]}, основателя стереохимии. Теперь сделаем небольшое отступление и скажем, что сейчас достаточно несложно понять откуда пошло название atom-bound connectivity index, ведь это есть хорошая попытка уместить в четырех словах тот смысл, что мы используем только структурную информацию о молекуле (atom and bounds), и при этом сконструировав граф отображаем его в одномерное векторное пространство (index). Теперь имея в представлении некий геометрический объект, мы можем изучать такие свойства как, например, межатомное расстояние, углы которые образуют связи между атомами и им подобные, которые зависят от конкретного расположения молекулы в пространстве. Кроме того, мы имеем возможность измерить эти показатели с достаточной точностью или, возможно, известны некие априорные знания об их значениях. Однако, уже стает ясно, что такие свойства молекулы обладают достаточно большой вариативностью, причиной этому есть, то что атомы внутри молекулы поддаются некому воздействию (то есть не пребывают в статическом состоянии), что в свою очередь порождает различные внутримолекулярные движения. Даже если пренебречь этими атомными движениями, геометрия молекулы в некоторой степени зависит от ее окружения (например, давления в случае кристаллической решетки и тому подобные). Трудности, связанные с приведением концепций молекулярной геометрии к принципам квантовой теории, так же имеет право быть изученными, и рассматриваются, например в \hyperlink{must_molecula_have_shape}{[7]}. 

\newpage

С другой стороны, не подлежит сомнению, что молекула сохраняет свою идентичность независимо от всех этих геометрических изменений. Следовательно, в молекуле должно быть что-то, что остается инвариантным при (незначительных) изменениях в ее геометрии. Беря во внимание все вышеперечисленное, неудивительно, что знания точных геометрических параметров молекул обычно не имеет большого значения, когда приходиться решать проблемы, которые возникают в задачах прикладной химии.
В реальности, химикам приходится довольствоваться гораздо менее подробной информацией о молекуле, в частности той, что мы ранее называли структурой молекулы (или молекулярная связность).

Далее еще раз представим структурное описание молекулы. Интуитивно понятно, что исключительным в молекулах (инвариантом) является не длина атомных связей или углы между ними, а, скорее, представление как некого целого объекта в трехмерном пространстве. 



\newpage

\subsection{Химия и теория графов}
\subsection{Основы теории графов}
\subsection{Отношения}

Пусть $V$ представляет из себя множество некоторых элементов, обозначим их как  $\{ v_1, v_2, \dots, v_n \}$. В дальнейшем мы всегда будем считать, что множество является конечным, то есть содержит в себе конечное число элементов.

Множество $V \times V$ содержит в себе все упорядоченные $(v_r, v_s)$ состоящие из элементов множества $V$, то есть
$V \times V = \{(v_r, v_s) : r, s \in \{1, 2, \dots, n \} \}$. 
Любое подмножествов $R$ множества $V \times V$ называется отношение на множестве $V$. 

Например, $\{ (v_3, v_4), (v_1, v_5), (v_1, v_3) \}$ является отношение на множестве $\{ v_1, v_2, v_3, v_4, v_5  \}$.

Естественным образом, отношение можно представить графически, ассоциировал объекты множества, например, с точками на плоскости, а связи между объектами - оринтируемыми линиями между точками. Например, ниже изобразим графическое представление отношения, которое выше было рассмотрено как пример.

\begin{figure}[h]
\includegraphics[scale=0.3]{relation_example}
\centering
\caption{Графическое изображение отношения}
\end{figure}


Пусть $R$ является отношением на множестве $V$, и $u$ и $v$ принадлежать множеству $V$. Тогда отношение $R$ называется симметричным, если из того, что $(v_u, v_r) \in R$ следует, что  $(v_r, v_u) \in R$. Отношение $R$ называется рефлексивным, если $\forall v \in V: (v, v) \in R$. Отношение называется антирефлексивным, если $\forall v \in V: (v, v) \notin R$.

Например, отношение $\{ (v_2, v_5), (v_4, v_5), (v_5, v_2), (v_5, v_4) \}$ является симметричным и антирефлексивным на множестве $\{ v_1, v_2, v_3, v_4, v_5  \}$ и имеет следующее графическое представление.

\begin{figure}[h]
\includegraphics[scale=0.20]{relation_example_two}
\centering
\caption{Графическое изображение отношения}
\end{figure}

Если отношение является симметричным и антирефлексивым, то всегда двум ориентированным парам $(v_r, v_s)$ и $(v_s, v_r)$, мы однозначно можем поставить в соответствие неориентированную пару $\{v_r, v_s\}$. Следовательно, симметричное и антирефлексивное отношение на множестве $V$ - это набор некотрых неупорядоченных пар элементов множества $V$. В графическом представлении отношения, пара противоположной направленных линий может быть заменена неориентированной линией.

Отношение, которые мы рассмотрели в предыдущем примере, теперь может быть записано следующим образом $\{ \{v_2, v_5 \}, \{ v_4, v_5 \}  \}$, и ниже приведем соотвествующую ему диаграмму.

\begin{figure}[h]
\includegraphics[scale=0.30]{unordered_relation}
\centering
\caption{Графическое изображение симметричного и антирефлексивного отношения}
\end{figure}

\subsection{Определение графа, способ I}

Рассмотрим конечное множество $V$. Пускай $E$ представляет из себя симметричное и антирефлексивное отношение на $V$, то есть $E$ это множество неупорядоченных пар из элементов множества $V$. Тогда говорят, что множество $V$ с отношение $E$ образует граф. Если мы обозначим граф через $G$, тогда пишут $G = (V; E)$.

В сущности своей, последний пример на отношения, который мы рассмотрели, то есть множество $V = \{ v_1, v_2, v_3, v_4, v_5 \} $ и отношение на этом множестве $E =\{ \{v_2, v_5  \} \{v_4, v_5 \} \}$, уже представляют из себя граф.  

Процедура графического представления графа $G=(V, E)$ является предельно очевидной: с каждый элементом из множества $V$ мы ассоциируем точку (или маленький круг), а каждому элементу из множества $E$ ставим в соответствие неоринтируванную линию, которая соединяет точки.

Элементам множества $V$ называют вершинами графа $G$ (иногда можно встретить название узлы, точки). Элементы отношения $E$, которое содержит неупорядоченные пары, называются ребрами (иногда линиями).

Даное определение графа может быть без проблем расширено на тот случай, если отношение $E$ не является симметричным (тогда речь идет о ориентированных графах), или на случай, когда отношение не является антирефлексивным (граф содержит петли, то есть ребра из вершины в саму себя), или на оба случая сразу. Нас не будут интересовать эти типы графов, так как для наших нужд (иссследование структуры молекулы) вполне достаточной неориентированных графов без петель.

Графы, которые мы определили в данном разделе, называются простыми. Следовательно, простой граф не имеет ориентированных ребер и/или которые начинаются и заканчиваются в одной вершины (петли). Так же простой граф не может содержать кратных ребер (это непосредственно следует из определения множества и невозможности содержать в нем дубли как таковые).
 
\subsection{Определение графа, способ II}

Пусть $V = \{ v_1, v_2, \dots, v_n \}$ и $E = \{ e_1, e_2, \dots, e_m \}$ представляют из себя два конечных множества. Пусть $f$ является отображением, которое ставит в соответствие каждому элементу множества $E$ неупорядеченную пару из элементов множества $V$. Следовательно, для любого $e_i \in E$ существует такая уникальная (то есть отображение являетcя инъективным) пара  $\{v_{i_1}, v_{i_2} \}, \newline v_{i_1} \in V, v_{i_2} \in V$, такая что:

$$ f: e_i \rightarrow \{ v_{i_1}, v_{i_2} \} $$

Тогда два множества $V$ и $E$, вместе с отображением $f$ формируют граф. Этот граф обозначается как $(V, E, f)$.

Например, граф который фигурировал в предыдущих примерах, может быть определени с помощью множества $V = \{ v_1, v_2, v_3, v_4, v_5 \}$, множества 
$E = \{ e_1, e_2 \}$ и отображения $f$, которое задано как 

$$ f: e_1 \rightarrow \{ v_2, v_5 \}$$ 
$$e_2 \rightarrow \{ v_4, v_5 \} $$

Преимущество этого определения заключается в том, что оно может быть без проблем расширено для мультиграфов (графы, которые имеют кратные ребра).
И вправду, нам не обязательно требовать, что бы отображение являлось инъективным; несколько элементов из множества $E$ могут быть отображены в одну и ту
же пару элементов из множества $V$.

К примеру, если $V = \{ v_1, v_2, v_3, v_4 \}$ и $E = \{ e_1, e_2, e_3, e_4, e_5 \}$ и
$$ f: e_1 \rightarrow \{ v_1, v_2 \} $$
$$ e_2 \rightarrow \{ v_2, v_3 \} $$
$$ e_3 \rightarrow \{ v_2, v_3 \} $$
$$ e_4 \rightarrow \{ v_2, v_3 \} $$
$$ e_5 \rightarrow \{ v_3, v_4 \} $$

тогда мы получаем мультиграф, который имеет следующие графическое представление.

\begin{figure}[h]
\includegraphics[scale=0.40]{multigraph}
\centering
\caption{Графическое представление мультиграфа}
\end{figure}

\newpage

Далее мы не будем рассматривать теорию мультиграфов, так как ранее мы уже отметили, что для наших целей (исследование структурной состовляющей молекулы) достаточно неориентированных графов без петель. Однако отметим, что естественным образом возникает следующая интерпретация мультиребр, если мы с простым ребром (две вершины соединяет ровно одно ребро) начинаем ассоциировать некоторый единичный вес, а всем мульти ребрам будем приписывать некорый вес больше единицы. Так же мы будем требовать условие, что вес на мульти ребрах будет вести себя как монотонно возрастающая функция от порядка мультиребра (количество ребр между двумя фиксированными вершинами, которые соединяет мульти ребро). То есть мы пришли к понятию взвешенного графа, говоря более строго, мы рассматриваем некоторую функцию $f_w$ (задает веса на всех ребрах), которая действует из множества ребер $E$ в $\mathbb{R}$, и далее, ориентируемым графом будем называть следующая четверку $(V, E, f, f_w)$.

\subsection{Вершины и ребра}

Если $G=(V, E)$ является графов, то элементы множества $V$ называются вершинами, а элементы множества $E$ ребрами. Далее, договоримся, что количество ребер мы будем обозначать $m$ ($|E| = m$), а количество вершин $n$ ($|V| = n$).

Для того, что бы подчеркнуть принадлежность множества вершин $V$ некоторому графу $G$, будем использовать обозначение $V(G)$.

Две вершины $u \in V$ и $v \in V$ называются смежными, если $\{u, v\} \in E$. Так же говорят, что $u$ и $v$ являются соседними вершинами, или, что вершины соединены ребром $e=\{ u, v \}$, или, что $u$ и $v$ являются концами ребра $e$, либо что ребро $e$ является инцедентным к вершинам $u$ и $v$. 

Количество вершин, которые являются смежными для фиксированной вершины $v$, называются степенью этой вершины (или валентностью, если мы подразумеваем под вершиной атом молекулы). Вершина которая имеет степень 0 называется изолированной. Вершина которая имеет степень 1 называется висячей (концевой).

\newpage
Например, рассмотрим граф $G_1$.

\begin{figure}[h]
\includegraphics[scale=0.40]{graph_g1}
\centering
\caption{Граф $G_1$}
\end{figure}


Граф $G_1$ имеет три изолированные вершины и четыре висячие вершины. Кроме того, граф $G_1$ имеет три вершины со степенью 2, одну вершину со степенью 3 и одну вершину со степенью 5.

Обозначим через $d_{v}$ степень вершины $v$, тогда выполняется следующее равенство.

$$ \sum_{v \in V} d_{v} = 2 m, $$

где $m$ представляет собой количество ребер графа (как ранее и оговаривалось). 
Выше приведенное равенство известно под названием, как лемма о рукопожатиях \hyperlink{handshaking_lemma}{[9]}. Отклоняясь к химическому подтексту, скажем, что из этой леммы следует, что количество атомов в молекуле, которые имеют нечетную валентность должно быть четно.

Последовательность вершин $v_{i_0}, v_{i_1}, \dots, v_{i_l}$ графа, таких что $v_{i_{j - 1}}$ и $v_{i_j}$ являются смежными для $j = 1, 2, \dots, l$, при этом каждая вершина встречается не более одного раза, называется путем или простым путем в графе $G$, который соединяет вершины $v_{i_0}$ и $v_{i_l}$. Длина этого пути равняется $l$, по той причине, что путь проходит через $l$ ребр.

Если между двумя вершинами графа существует простой путь, будем говорить, что вершины лежат в одной компоненте связности графа. В противном случае, вершины лежат в разных.

Будем обозначать число компонент связности графа $G$ через $k = k(G)$.
Например, для графа $G_1$, который мы рассмотрели в предыдущем примере, имеем, что $k(G_1) = 5$.

Граф для которого $k = 1$ называется связным.

Если $G_1, G_2, \dots, G_k$ являются компонентами связности графа $G$, то сам граф $G$ представляется в виде:

$$ G = G_1 \cup G_2 \cup \dots \cup G_k $$

Для вершин $u$ и $v$, которые лежат в одной компоненте связности, можно определить расстояние между ними, обозначим его $d(u, v)$, которое определяется как длина самого короткого пути между вершинами $u$ и $v$.

\newpage

\subsection{Изоморфизм графов}

Пусть $G$ и $H$ два графа, чьи наборы вершины $u_1, u_2, \dots, u_n$ и $v_1, v_2, \dots, v_n$, соответсвенно. Пусть $\pi$ является перестановкой чисел $1, 2, \dots, n$:

$$
\pi = 
\left(\begin{array}{cccc}
1 & 2 & \dots & n \\
\pi(1) & \pi(2) & \dots & \pi(n)
\end{array}\right)
$$

Предположим теперь, что может быть найдена такая перестановка $pi$, что для $i = 1, 2, \dots, n$ и $j = 1, 2, \dots, n$ вершины $u_i$ и $v_i$ являются смежными в графе $G$ тогда и только тогда, когда вершины $v_{n_i}$ и $v_{n_j}$ являются смежными в $H$. Тогда говорят, что два графа являются изоморфными, обычно обозначают как $G \simeq H$. 

В этом случае перестановка $\pi$ представляет собой изоморфное отображение из множества вершин $G$ в множество вершин $H$. Изоморфные отображения всегда сохраняют отношение смежности вершин.

Изоморфные графы концептуально представляют один и тот же математический объект. Различие заключается в только в том каким образом размечать вершины.

Изоморфное отображение из вершин графа в сами же вершины графа (то есть отображние сохраняет отношение смежности) называется автоморфизмом графа. Очевидно, каждый граф имеет тривиальный автоморфизм, или так называемый тождественный автоморфизм:

$$
\pi_0 = 
\left(\begin{array}{cccc}
1 & 2 & \dots & n \\
1 & 2 & \dots & n
\end{array}\right)
$$

Так же очевидно, что некоторые графы могут обладать неидентичным автоморфизмом. Например, рассмотрим граф $G_2$.

\begin{figure}[h]
\includegraphics[scale=0.30]{graph_g2}
\centering
\caption{Граф $G_2$}
\end{figure}

Граф $G_2$ имеет следующий автоморфизм:

$$
\pi_1 = 
\left(\begin{array}{cccccc}
1 & 2 & 3 & 4 & 5 & 6 \\
1 & 6 & 5 & 4 & 3 & 2
\end{array}\right)
$$

\newpage

Множество всех автоморфизмов графа образует группу. Количество автоморфизмов (порядок группы) играет ту важну роль, что мы можем посчитать количество изоморфный копий данного графа, а именно имеем следующие утверждение.

\textbf{Утверждение} \hyperlink{orbit_theorem}{[10]}. Пусть $G$ является графов на $n$ вершинах, $a$ - количество автоморфизмов этого графа. Тогда граф $G$ имеет $\frac{n!}{a}$ изоморфных копий. 

Утверждение непосредственно следует из Orbit-stabilizer theorem.

Количество изоморфным копий говорит о уровне самоподобии графа (или его симметрии). Так полностью ассиметричный в некором смысле граф имеет в точности $n!$ изоморфным копий, и при этом имеет один тривиальный автоморфизм. То есть мы можем смотреть на количество автоморфизмов как на показатель симметричность графов (и, например, введя таким образом порядок на графам можем ранжировать их по подобности).

Значение $I(G)$ которое мы будем ассоциировать с графом $G$, и которое имеет одинаковое значение для всех изоморфных копий графа $G$, называется инвариантом графа. Следовательно, инварианты графа - это величины, которые зависят только от структуры графа (то есть маркировка вершин графа не имеет значение).

\subsection{Специальные виды графов}

Если $E$ является пустым множеством, $E = \emptyset$, то тогда граф $G = (V, E)$ содержит $n$ изолированных вершин. Он имеет название пустой пустой граф (иногда нуль-граф). Если $E$ содержит все возможные пары элементов множества $V$, тогда соответствующий граф называется полным и обозначается как $K_n$. Первые пяти полных графов имеет следующие графическое изображение.


\begin{figure}[h]
\includegraphics[scale=0.30]{full_five}
\centering
\caption{Первые пять полных графов}
\end{figure}


Очевидно, что все вершины в графе $K_n$ являются смежными и расстояние между ними равно единице.

Граф, в котором все степени имеют одинаковую степень, обозначим ее через $g$, называется регулярным графов степени $g$. Согласно этому определению $K_n$ является регулярным графов степени $n-1$. В химических приложениях отдельную роль играют связные регулярные графы степени 2. Они имеет называние циклы (или контуры) и обозначаются как $C_n$. Первые пять контуров имеют следующий вид:


\begin{figure}[h]
\includegraphics[scale=0.30]{circuit_graph}
\centering
\caption{Первые пять графов контуров}
\end{figure}



\newpage

\subsection{Специальные молекулярные графы} 

\subsection{Ациклические молекулы}

Как известно, в теории графов, ациклический связный граф (напомним, что структуру графа описывают только неориентированные графы) называется деревом. Следовательно, мы можем сказать, что топология ациклических молекул может быть представлена деревьями. Заметим, что структурный граф молекулы всегда является связным, в противном случае, существовали бы изолированные атомы, которые не имеют никаких связей с атомами молекулы, и как следствие нет никаких предпосылок относить этот атом к молекуле, что и потверждает наши рассуждения о связности структурного графа молекулы. Помимо этого, утверджается, что основные топологические свойства ациклических молекул совпадают со свойствами деревьев. Ниже мы рассмотрим основные свойства деревьев.

\subsection{Деревья}

В некотором смысле деревья представляют из себя самый "простой" класс графов, то есть, тот который лучше других поддается исследованиям, и многих задачи которые не решены в общем случае (то есть для произвольного графа) имееют решение для деревьев. Ниже представим все деревья, которые имеют не более чем шесть вершин.


\begin{figure}[h]
\includegraphics[scale=0.3]{all_tree}
\centering
\caption{Все деревья, которые имеет шесть либо меньше вершин}
\end{figure}

\newpage

Заметим, что существует в точности одно дерево на 1, 2 и 3 вершина. При $n \geq 4$ существует несколько деревьев с $n$ вершинами, при чем их количество быстро растет с ростом $n$. (ознакомится с последовательность количество деревьев на $n$ вершинах можно тут \hyperlink{count_tree}{[11]})

Приведем следующую теорему, которая помогает охарактеризовать деревью.

Сначала, докажем одну вспомогательную лемму.

\textbf{Лемма.} Пусть существует произвольный путь между двумя вершинами графа, тогда между этими вершинами существует простой путь.



\textbf{Теорема}. Пусть $G$ является неоринтированным графов, который содержит $n$ вершин и $m$ ребер. Тогда утверждения $(1) - (5)$ являются эквивалентными.

$(1)$ $G$ является деревом, то есть $G$ связен и не содержит циклов.

$(2)$ $G$ является ациклическим и содержит ребер на одно меньше, чем вершин (то есть $m = n - 1$).

$(3)$ $G$ является связным и содержит ребер на одно меньше, чем вершин (то есть $m = n - 1$).

$(4)$ $G$ является связным, но если удалить любое ребро из графа, граф перестает быть связным.

$(5)$ $G$ является ациклическим, и любой граф $G'$, полученный путем добавление ребер в граф $G$, содержит цикл.

\textbf{Доказательство.}

$(1) \Rightarrow (2)$.

Ацикличность получаем из определения дерева. 

\newpage  

\section{ABC индекс как полный инвариант графа}

Ранее мы задавались вопросом является ли ABC полным инвариантом графа (нет, нашли контрпример), то есть позволяет установить факт изоморфности графа основываясь исключительно на значениях индекса. Так как изначальная задача состоит не в отыскании наиболее полного графового "инварианта", а в исследовании свойст уже выбранного индекса, то наймемся поиском классов графов на которых ABC индекс имеет свойства полного графового инваринта.

\textbf{Теорема.} Для множества деревьев, которые имеют не более чем 6 вершин, ABC индекс является полным инвариантом.


\textbf{Доказательство.}
Доказательство будет проводить непосредственно рассмотрев все 14 возможных неизоморфным друг другу деревьев, которые имеет не более чем 6 вершин.

Рассмотрим дерево $T_2$.

\begin{figure}[h]
\includegraphics[scale=0.5]{T2}
\centering
\caption{Дерево T2}
\end{figure}

Посчитаем ABC индекс для дерева $T_2$.

$$ ABC(T_2) = \sqrt{\frac{1 + 1  - 2} {1 * 1}} = 0 $$

Далее рассмотрим дерево $T_3$.

\begin{figure}[h]
\includegraphics[scale=0.5]{T3}
\centering
\caption{Дерево T3}
\end{figure}

Посчитаем ABC индекс для дерева $T_3$.

$$ ABC(T_3) = \sum_{uv \in E(G)} \sqrt{\frac{d_u + d_v + 2}{d_u d_v}} =  2 \frac{1}{\sqrt{2}} = \sqrt{2} $$


Рассмотрим дерево $T_4$.

\begin{figure}[h]
\includegraphics[scale=0.5]{T4}
\centering
\caption{Дерево T4}
\end{figure}

Посчитаем ABC индекс для дерева $T_4$.

$$ ABC(T_4) = \sum_{uv \in E(G)} \sqrt{\frac{d_u + d_v + 2}{d_u d_v}} =  2 \frac{1}{\sqrt{2}} + \frac{1}{\sqrt{2}} = \frac{3}{\sqrt{2}}
= \frac{3 \sqrt{2}}{2} $$


Далее рассмотрим дерево $T_5$.

\begin{figure}[h]
\includegraphics[scale=0.5]{T5}
\centering
\caption{Дерево T5}
\end{figure}

Посчитаем ABC индекс для дерева $T_5$.

$$ ABC(T_5) = \sum_{uv \in E(G)} \sqrt{\frac{d_u + d_v + 2}{d_u d_v}} = 3 \sqrt{\frac{2}{3}} = 3\frac{\sqrt{6}}{3} = \sqrt{6} $$

Рассмотрим дерево $T_6$.

\begin{figure}[h]
\includegraphics[scale=0.5]{T6}
\centering
\caption{Дерево T6}
\end{figure}

Посчитаем ABC индекс для дерева $T_6$.

$$ ABC(T_6) = \sum_{uv \in E(G)} \sqrt{\frac{d_u + d_v + 2}{d_u d_v}} = 4 \frac{1}{\sqrt{2}} = 2 \sqrt{2} $$

Далее рассмотрим дерево $T_7$.

\begin{figure}[h]
\includegraphics[scale=0.5]{T7}
\centering
\caption{Дерево T7}
\end{figure}

Посчитаем ABC индекс для дерева $T_7$.

$$ ABC(T_7) = \sum_{uv \in E(G)} \sqrt{\frac{d_u + d_v + 2}{d_u d_v}} = 2 \sqrt{\frac{2}{3}} + 2 \frac{1}{\sqrt{2}} = \frac{2\sqrt{6} + 3 \sqrt{2}}{3} $$


Рассмотрим дерево $T_8$.

\begin{figure}[h]
\includegraphics[scale=0.5]{T8}
\centering
\caption{Дерево T8}
\end{figure}

Посчитаем ABC индекс для дерева $T_8$.

$$ ABC(T_8) = \sum_{uv \in E(G)} \sqrt{\frac{d_u + d_v + 2}{d_u d_v}} = 4 \frac{\sqrt{3}}{2} = 2 \sqrt{3}  $$


Далее рассмотрим дерево $T_9$.

\begin{figure}[h]
\includegraphics[scale=0.5]{T9}
\centering
\caption{Дерево T9}
\end{figure}

Посчитаем ABC индекс для дерева $T_9$.

$$ ABC(T_9) = \sum_{uv \in E(G)} \sqrt{\frac{d_u + d_v + 2}{d_u d_v}} = 5 \frac{1}{\sqrt{2}} = \frac{5}{2} \sqrt{2}  $$



Рассмотрим дерево $T_{10}$.

\begin{figure}[h]
\includegraphics[scale=0.5]{T10}
\centering
\caption{Дерево T10}
\end{figure}

Посчитаем ABC индекс для дерева $T_{10}$.

$$ ABC(T_{10}) = \sum_{uv \in E(G)} \sqrt{\frac{d_u + d_v + 2}{d_u d_v}} = 2 \sqrt{\frac{2}{3}} + 3 \frac{1}{\sqrt{2}} =
\frac{4 \sqrt{6} + 9 \sqrt{2}}{6} $$


Далее рассмотрим дерево $T_{11}$.

\begin{figure}[h]
\includegraphics[scale=0.5]{T11}
\centering
\caption{Дерево T11}
\end{figure}

Посчитаем ABC индекс для дерева $T_{11}$.

$$ ABC(T_{11}) = \sum_{uv \in E(G)} \sqrt{\frac{d_u + d_v + 2}{d_u d_v}} = 4 \frac{\sqrt{2}}{2} + \sqrt{\frac{2}{3}} = \frac{6 \sqrt{2} + \sqrt{6}}{3}$$


Рассмотрим дерево $T_{12}$.

\begin{figure}[h]
\includegraphics[scale=0.5]{T12}
\centering
\caption{Дерево T12}
\end{figure}

Посчитаем ABC индекс для дерева $T_{12}$.

$$ ABC(T_{11}) = \sum_{uv \in E(G)} \sqrt{\frac{d_u + d_v + 2}{d_u d_v}} = 3 \frac{\sqrt{3}}{2} + 2 \frac{\sqrt{2}}{2} = \frac{3 \sqrt{3} + 2 \sqrt{2}}{2}$$

\newpage

Далее рассмотрим дерево $T_{13}$.

\begin{figure}[h]
\includegraphics[scale=0.5]{T13}
\centering
\caption{Дерево T13}
\end{figure}

Посчитаем ABC индекс для дерева $T_{13}$.

$$ ABC(T_{13}) = \sum_{uv \in E(G)} \sqrt{\frac{d_u + d_v + 2}{d_u d_v}} = 4 \frac{\sqrt{6}}{3} + \frac{2}{3} = \frac{4 \sqrt{6} + 2}{3} $$


Рассмотрим дерево $T_{14}$.

\begin{figure}[h]
\includegraphics[scale=0.5]{T14}
\centering
\caption{Дерево T14}
\end{figure}

Посчитаем ABC индекс для дерева $T_{14}$.

$$ ABC(T_{14}) = \sum_{uv \in E(G)} \sqrt{\frac{d_u + d_v + 2}{d_u d_v}} = 5 \sqrt{\frac{4}{5}} = 2 \sqrt{5} $$

Как видим выше, для каждого дерева значение ABC индекса является уникальным, то есть его знание позволяет установить факт изоморфизма двух деревьев, которые имеют не более чем $6$ вершин.

\bigskip

\textbf{Теорема} Для множества деревьев, которые имеют ровно 7 вершин ABC индекс \textbf{не} является полным графовым инвариантом.

\textbf{Доказательство.}

В качестве доказательства привидем контрпример, а именно два дерева которые имеют одинаковый ABC индекс, но при этом не являются изоморфными.  

\newpage

Рассмотрим, первое дерево $P_1$.

\begin{figure}[h]
\includegraphics[scale=0.5]{P1}
\centering
\caption{Дерево P1}
\end{figure}

Посчитаем для него ABC индекс.

$$ ABC(P_1) = 6 \frac{\sqrt{2}}{2} = 3 \sqrt{2} $$

Далее рассмотрим дерево $P_2$.

\begin{figure}[h]
\includegraphics[scale=0.5]{P2}
\centering
\caption{Дерево P2}
\end{figure}

Посчитаем для него ABC индекс.

$$ ABC(P_2) = 6 \frac{\sqrt{2}}{2} = 3 \sqrt{2} $$

Очевидно, что деревья $P_1$ и $P_2$ не являются изоморфными, но при этом имеют одинаковый ABC, что и доказывает теорему.

\bigskip

Заметим, так как вычисление ABC индекса использует только информацию о степенях вершин графа, то тогда существует некая степенная характеристика (и в тоже время инвариант) графа, которая вырождает ABC индекс на всех графах, для которых эта характеристика совпадает. Далее разберемся, что представляет из себя эта степенная характеристика.

Сначала покажем, что при совпадении такого инварианта, как отсортированный список степеней графа, ABC индекс не вырождается, то есть может принимать различные значение на графах которые не являются изоморфными и при этом имеют одинаковый отсортированный список степеней. Для этого приведем контрпример двух графов, которые имеют одинаковый отсортированный список степеней графа, но при этом имееют различные ABC индексы.

\newpage

Рассмотрим граф $G_1$. 

\begin{figure}[h]
\includegraphics[scale=0.5]{G1}
\centering
\caption{Граф G1}
\end{figure}

Посчитаем для него ABC индекс.

$$ ABC(G_1) = \sqrt{\frac{2}{3}} + 6 \frac{\sqrt{2}}{2} = \frac{\sqrt{6} + 9 \sqrt{2}}{3} $$

Далее рассмотрим граф $G_2$.

\begin{figure}[h]
\includegraphics[scale=0.5]{G2}
\centering
\caption{Граф G2}
\end{figure}

$$ ABC(G_2) = 2\sqrt{\frac{2}{3}} + 5 \frac{\sqrt{2}}{2} = \frac{4 \sqrt{6} + 15 \sqrt{2}}{6} $$

Как видим выше оба графа $G_1$ и $G_2$ имееют одинаковый отсортированный список степеней вершин, а именно $(3, 2, 2, 2, 2, 1, 1, 1)$, при этом графы не являются изоморфными, и так же имеют различный ABC индекс, что доказывает то, что данный инвариант не вырождает ABC индекс.

\bigskip

Далее введем такой инвариант графа, как отсортированный список пар степеней ребер. Приведем его формальное описание, пусть $E = \{e_1, e_2, \dots, e_n \}$ является множеством ребер некоторого графа $G$, с каждым ребром $e_{i} = \{ u_{e_{i}}, v_{e_{i}} \}$ (где через $u_{e_{i}}$ и $v_{e_{i}}$ обозначим вершины которые являются инцендентными данному ребру) будем ассоциировать пару $(\min(d_{u_{e_{i}}}, d_{v_{e_{i}}}), \max(d_{u_{e_{i}}}, d_{v_{e_{i}}}))$, то есть упорядоченную пару степеней вершин, которые инцидентные заданному ребру. Теперь объединяя в список все такие пары и упорядочив его (если первые элементы пары различны, то сравниваем их, иначе сравниваем вторые элементы пары) получаем отсортированный список пар степеней ребер.

\newpage

Покажем на примере как выглядит отсортированный список пар степей ребер. Рассмотрим граф $G$.


\begin{figure}[h]
\includegraphics[scale=0.5]{graph_g}
\centering
\caption{Граф G}
\end{figure}

Граф $G$ имеет следующий отсортированный список пар степеней ребер: $((1, 3), (1, 3), (2, 2), (2, 3), (2, 3), (3, 3))$.

Очевидно, ABC индекс будет вырождаться на множествах графов, которые имеют одинаковый отсортированный список пар степеней ребер, то есть мы не можем различивать изоморфизм на графах которые являются неизоморфными и при этом имеют одинаковый отсортированнный список пар степеней ребер. 

Как пример, приведем два неизоморфным графа, которые имеют одинаковый отсортированный список пар степеней ребер (и как следствие одинаковый ABC индекс).

Граф $G_1$.

\begin{figure}[h]
\includegraphics[scale=0.5]{second_g1}
\centering
\caption{Граф G1}
\end{figure}

Граф $G_2$

\begin{figure}[h]
\includegraphics[scale=0.5]{second_g2}
\centering
\caption{Граф G2}
\end{figure}

Как видим выше, оба графа $G_1$ и $G_2$ имеют одинаковый одинаковый отсортированный список пар степеней ребер, а именно $((1, 2), (2, 2), (2, 2), (2, 3), (2, 3), (2, 3))$, и при этом имеют ABC индекс равный $3 \sqrt{2}$. 

Обозначим через $\mathfrak{G}_n$ множество всех связных графов на $n$ вершинах, которые являются попарно неизоморфными (количество таких графов на $n$ вершинах образуют следующую последовательность \hyperlink{count_graphs}{[12]}). 

Тогда множество всех связных графов, которые являются попарно неизоморфными, запишем как $\mathfrak{G} = \bigcup\limits_{k=1}^{+\infty} \mathfrak{G}_n$.

Так как нас интересуют только те множества графов, на которых ABC индекс является полным инвариантом, то для нас не является приемлемой ситуации, когда в одно множество попадают два графа, которые имееют одинакой отсортированный список пар степеней ребер. Для решения этой проблемы, посмотрим на равенство графов, как на равенство их отсортированных списков пар степеней ребер. Семантически равенство образует отношение эквивалентности (то есть является рефлексивным, симметричным и транзитивным). Далее разбиваем все множество графов $\mathfrak{G}$ на классы эквивалентности в смысле равенстве отсортированных списков пар степеней ребер. Затем будем формировать множества $\mathfrak{G}^{\gamma}$ $\forall \gamma \in \Gamma$. Где $\gamma$ представляет из себя некоторый индекс, который характеризует уникальный выбор выбор представителей из каждого классы эквивалентности, то есть $\gamma$ отчевает за то, какой именно элемент выбирается из каждого класса. Заметим, что мы выбираем ровно одного представителя из класса эквивалентности (иначе мы получаем нежелательную ситуацию, с которой мы изначально боремся), и обязательно выбираем из всех классов эквивалентности, иначе множество можно было сделать более "широким". $\Gamma$ представляет из меня множество все возможных выборов $\gamma$. Далее чтобы получать множества на которых ABC индекс является полным графовым инвариантом, будем пересекать предлагаемые множества с каждый множестов $\mathfrak{G}^{\gamma} \forall \gamma \in \Gamma$. 

Далее рассмотрим графы, которые имеют степень не более 4 (так называемые химические графы). Так как ABC индекс оперирует только со степенями вершин, мы можем рассмотреть таблицу, где приведена зависимость значения $\sqrt{\frac{d_1 + d_2 - 2}{d_1 d_2}}$ от соответствующих степеней вершин инцидентных некоторому
ребру. Заметим, что значение для различных пар степеней могут совпадать.


\begin{table}[ht]
\centering
\begin{tabular}{|c|c|c|c|c|}
\hline
I / II  & 1                    & 2                    & 3                    & 4                     \\ \hline
1                                               & 0                    & $\frac{1}{\sqrt{2}}$ & $\sqrt{\frac{2}{3}}$ & $\frac{\sqrt{3}}{2}$  \\ \hline
2                                               & $\frac{1}{\sqrt{2}}$ & $\frac{1}{\sqrt{2}}$ & $\frac{1}{\sqrt{2}}$ & $\frac{1}{\sqrt{2}}$  \\ \hline
3                                               & $\sqrt{\frac{2}{3}}$ & $\frac{1}{\sqrt{2}}$ & $\frac{2}{3}$        & $\sqrt{\frac{5}{12}}$ \\ \hline
4                                               & $\frac{\sqrt{3}}{2}$ & $\frac{1}{\sqrt{2}}$ & $\sqrt{\frac{5}{12}}$ & $\sqrt{\frac{3}{8}}$  \\ \hline
\end{tabular}
\end{table}

\newpage

Таким образом для химического графа мы можем представить себя ABC как некую линейную комбинацию (с натуральными коэффициентами) элементов вышеприведенной матрицы.


Обозначим рассмотренную ранее таблицу (матрицу) как $C = (c_{ij}) \; i \in \{1, 2, 3, 4\}, j \in \{1, 2, 3, 4\}$. Теперь нас будет интересовать следующий вопрос (пока мы работаем только с графами, максимальная степень которых не превышает 4), а при каких условия ABC индексы для некоторых графов $G_1$ и $G_2$ будут совпадать только при совпадении самих графов $G_1$ и $G_2$. Также, тут мы считает, что графы $G_1$ и $G_2$ имеют различный отсортированный список пар степеней ребер.

$$ ABC(G_1) =  \sum_{i=1}^{4} \sum_{j=1}^{4} c_{ij} k_{ij}^{G_1} $$
$$ ABC(G_2) = \sum_{i=1}^{4} \sum_{j=1}^{4} c_{ij} k_{ij}^{G_2} $$

Где $k_{ij}^{G_1}$ представляет из себя количество ребер в графе $G_1$,	для которых степени инцидентных вершин равны $i$ и $j$ соответственно. $k_{ij}^{G_2}$ представляет из себя величину аналогичную $k_{ij}^{G_2}$, но только для графа $G_2$. 

То есть, опираясь на ранее сказанное, нас будут интересовать ситуация, в которых ABC индекс совдает только при полном соответствии количества различных ребер в графах. Другими словами мы ищем условия на графы $G_1$ и $G_2$, при которых будет верно следующее утверждение 
$ G_1 = G_2  \iff ABC(G_1) = ABC(G_2) \iff k_{ij}^{G_1} = k_{ij}^{G_2} \; \forall i \in \{ 1, 2, 3, 4 \} \; \forall j \in \{ 1, 2, 3, 4 \} $. 

Первой из потенциально неблагоприятных ситуаций для нас являтся вырождение функции $\sqrt{\frac{x + y - 2}{xy}}$ на некоторых наборах точек. Например, для пар степеней $(1, 2)$ и $(1, 3)$, имеем что $\sqrt{\frac{1 + 2 - 2}{1 * 2}} = \frac{1}{\sqrt{2}} = \sqrt{\frac{1 + 3 - 2}{1 * 3}}$. 

Более общо, пуcть для $(d_1, d_2)$ и $(d_3, d_4)$ функция $\sqrt{\frac{x + y - 2}{xy}}$ вырождается, то есть $\sqrt{\frac{d_1 + d_2 - 2}{d_1 d_2}} =
\sqrt{\frac{d_3 + d_4 - 2}{d_3 d_4}}$. Тогда применяя законы дистрибутивности, получаем, что в нашей сумме для ABC индекса графа $G_1$ образуется следующие слагаемое:
$$ \sqrt{\frac{d_1 + d_2 - 2}{d_1 d_2}} (k_{d_1 d_2}^{G_1} + k_{d_3 d_4}^{G_1})$$

И аналогично, в сумме ABC индекса для графа $G_2$ образуется слагаемое:

$$ \sqrt{\frac{d_1 + d_2 - 2}{d_1 d_2}} (k_{d_1 d_2}^{G_2} + k_{d_3 d_4}^{G_2}) $$


Очевидным образом, нам это говорит о том, что ABC индекс не сможет различать графы, для которых суммарное количество ребер (тут мы говорим о ребре как о паре $(d_u, d_v)$ вершин ему инцидентных) c равным значением функции $\sqrt{\frac{x + y - 2}{x y}}$, совпадает в графах $G_1$ и $G_2$, но при этом количество хотя бы одного ребра не является одинаковым.

То есть первым ограничением, которе мы наложим, будет запрет в графах ребер иметь более одного ребра, для которых функция $\sqrt{\frac{x + y - 2}{x y}}$ совпадет. Так как мы продолжаем формировать множества на которых ABC индекс являтся полным инвариантом, то мы проведем ранее описаную процедуру для разбиения множества всех графов $\mathfrak{G}$ на классы эквивалентности в смысле равенства как наличия в графе более двух ребер, которые имеют одинаковое значение функции $\sqrt{\frac{x + y - 2}{x y}}$, и далее сформируем все возможные множества, которые содержат по одному представителю из каждого класса эквивалентности.

С другой стороны такое ограничение можно несколько ослабить, так как наличие в графе хотя бы двух ребер на которых функция $\sqrt{\frac{x + y - 2}{x y}}$ вырождается, еще не ведет к тому, что ABC индекс будет не сможет различить графы с таким свойством. Например, пусть граф 

Перед тем как далее исследовать ABC индекс, нам придется ответить на следующий вопрос. Пусть мы имеем два вектора (которые состоят из рациональных чисел и имеют некоторую размерность $n$) обозначим их как $\vec{\textbf{x}} = (x_1, x_2, \dots, x_n)$ и $\vec{\textbf{b}} = (x_1, x_2, \dots, x_n)$ соответсвенно, и так же имеем некоторый произвольный вектор чисел $\vec{\textbf{z}} = (z_1, z_2, \dots, z_n)$, нас интересуют условия на $\vec{\textbf{z}}$ при которых равенство $<x; z> = <y; z>$ выполняется 

\newpage


\section{ABC индекс как эвристика для обнаружения изоморфизма графов}

Ранее мы показали (нашли контрпример) того, что ABC индекс не являтся полным графовым инвариантом, то есть существуют неизоморфные графы которые имеет одинакой ABC индекс. Но с другой стороны, мы можем утверждать что графы не являются изоморфными, если их ABC индексы различны. В этом и заключается основная идея эвристик: при проверке на изоморфизм двух графов, перед тем как использовать вычислительно затратные алгоритмы (и как нам известно, задача проверки на изоморфизм требует в худшем случае $O(|V| !)$ действий, $|V|$ - количество вершин), мы имеем возможность применить линейно вычислимые эвристики (или любые другие полиномиальные, что в любом случае асимптотически лучше экспоненциального алгоритма) и утверждать, что графы не являются изоморфными.

Так же ставят вопрос о том, что бы найти набор эвристик достаточный для установления факта изоморфизма двух графов. То есть мы последовательно применяем все эвристики, и только в том случае, если все эвристики говорят про факт изоморфности двух графов (помним, что каждая эвристики в отдельности может давать 
ложноположительные результаты), мы делаем заключение, что графы являются истинно изоморфными. 

Как и ранее мы будем работать с неориентированными графами без разметки (то есть, мы не отличаем изоморфные копии графа). Ранее мы говорили, что ABC индекс вырождается, если для некоторых двух различных графов $G_1$ и $G_2$, выполняется что $ABC(G_1) = ABC(G_2)$. Аналогично мы можем говорить для любого топологического индекса, то есть пусть $I: \mathfrak{G} \rightarrow \mathbb{R_{+}}$ является топологическим индексом (или другими словами некоторое отображение из пространства всех графов в некоторое число), тогда будем говорить $I$ вырождается на некотором множества графов, если существую два графа $G_1$ и $G_2$ принадлежащие этому множеству, такие что $I(G_1) = I(G_2)$.

Что бы оценивать качество эвристики, надо сначала определиться, что именно мы подразумеваем под качеством. Так как мы хотим научится определять степень вырожденности топологического индекса $I$, будем использовать для получения оценок множества $\mathfrak{G}_n$ (множество связных попарно неизоморфным графов на $n$ вершинах), где размер $n$ будем выбирать из практических соображений (скажем, $n \geq 9$). Нас интересует, сколько существует неупорядоченных пар графов $G_1$ и $G_2$ (при этом $G_1 \neq G_2$, и $G_1, G_2$ принадлежат $\mathfrak{G}_n$), такие что $I(G_1) = I(G_2)$. 

В данный момент вместо произвольного топологического индекса $I$ продолжим рассматривать ABC индекс, но при этом, все рассуждения будут оставаться верными и для произвольного топологического индекса $I$, кроме той детали, что асимптотика вычисление произвольного топологического индекса может быть отлична от линейной (ABC индекс линейно вычислим по числу ребер графа). Заметим, что $n^2$ является оценкой оценку сверху  на количество ребер в графе, так как, очевидно, что полный граф содержит $\frac{n(n-1)}{2}$ ребер и оценка является правильной. Теперь корректно говорить, что ABC индекс является линейно вычислимым по числу ребер (то есть, асимптотика $O(m)$), либо имеет квадратичную сложность по числу вершин (то есть, асимптотика $O(n ^ 2)$). 	 

Зафиксируем некоторое $n$ и соответственно множество графов $\mathfrak{G}_n$, обозначим через $N = |\mathfrak{G}_n|$, тогда общее число пар графов $C_{N}^{2} = \frac{N (N - 1)}{2}$. Перед тем как научиться быстро считать количество пар на которых ABC индекс вырождается рассмотрим тривиальный алгоритм. Очевидно, что мы можем перебрать все пары, получив суммарную асимпотику $O(\sum_{i=1}^{N-1}\sum_{j=i+1}^{N} m_i + m_j)$ = $O(N(m_1 + m_2 + \dots + m_N))$, где $m_1, m_2, \dots, m_{N}$ - количество ребер в каждом графе из множества $\mathfrak{G}_n$ соответственно, так как $O(N^2)$ мы тратим только на чтобы рассмотреть все пары и $O(m_i + m_j)$ на вычисление ABC индекса в каждой паре. На самом деле этот алгоритм можно немного улучшить, вместо того чтобы на каждой паре пересчитовать ABC индекс для обоих графов, мы предпосчитаем ABC индекс для каждого графа из $\mathfrak{G}_n$ (очевидно, так как множество является конечным, мы можем однозначно ассоциировать каждый граф с некоторым числом (его порядковым номером), это является необходимым условие, чтобы организовать эффективное отображение из множества графов в действительные числа, и в нашем случае это возможно за $O(1)$). То есть объединяя все прежде сказанное, мы тратим $O(m_1 + m_2 + \dots + m_N)$ на предпосчет ABC индекса и построения отображения, а далее затрачиваем $O(N^2)$ действий чтобы перебрать все пары, что в итоге дает нам $O(N^2 + m_1 + m_2 + \dots + m_N)$ действий.

Теперь предложим более эффективный алгоритм подсчета количества пар из $\mathfrak{G}_n$ на которых ABC вырождается. Очевидно, что множество $\mathfrak{G}_n$ является конечным, $\mathfrak{G}_n = \{ G_1, G_2, \dots, G_N \}$. Посчитает ABC индекса на всех элементах из этого множества и отсортируем полученные значения, обозначим их как $a_1, a_2, \dots, a_N$ $(a_1 \leq a_2 \leq \dots \leq a_N)$, очевидно что одинаковые значение ABC индекса будут находится рядом в отсортированной последовательности, что дает нам возможность возможность разбирать значения на группы с одинаковыми значениями за линейный проходный, нас на самом даже не будут интересовать сами группы, а только количество элементов в каждой из группы. Пусть мы получили $K$ различных значений ABC индеса на $N$ графах, и обозначим через $k_1, k_2, \dots, k_K$ количество элементов в каждой группе, тогда $N = k_1 + k_2 + \dots + k_K$. Нам остается посчитать сколько получается неупорядоченных пар на которых ABC индекс вырождается, понятно, что достаточно посчитать количество пар только в каждой из $K$ групп, так как различные группы имеют различное значение ABC индекса, тогда количество неупорядоченных пар в каждой групе будет $C_{k_i}^2 \; \forall i \in \{1, 2, \dots, K \}$, и очевидно, суммарное количество пар будет составлять $\sum_{i=1}^{K} C_{k_i}^2$. Для точности изложения добавим, что если $k_i, i \in \{1, 2, \dots, K \}$ меньше $2$, то значение $C_{k_i}^{2}, i \in \{ 1, 2, \dots, K \}$ считаем равным нулю. В результате мы получаем асимптотику равную $O(N log(N))$, так как мы тратим $O(N log(N))$ на сортировку, и потом за линейный проход по отсортированным значение делаем необходимый подсчет.


\newpage



\begin{thebibliography}{3}

\bibitem{abc_index implementation}
\hypertarget{first_bibitem}{}
\href{https://github.com/dertuner-flex/abc_index/}{ brutforce-implementation on github }

\bibitem{fisrt_mention ABC index}
\hypertarget{fisrt_mention}{}
E. Estrada, L. Torres, L. Rodríguez, I. Gutman, An atom-bond connectivity index: modelling the enthalpy of formation of alkanes, Indian J. Chem. 37A
(1998) 849–855.

\bibitem{QSAR link}
\hypertarget{qsar_link}{}
\href{https://en.wikipedia.org/wiki/Quantitative_structure%E2%80%93activity_relationship}{ Quantitative structure–activity relationship }


\bibitem{Anonimus random walk}
\hypertarget{random_walk}{}
\href{https://arxiv.org/pdf/1805.11921.pdf}{ Anonymous Walk Embeddings }

\bibitem{Molecular geometry}
\hypertarget{molecular_geometry}{}
\href{ https://en.wikipedia.org/wiki/Molecular_geometry }{ Molecular geometry }


\bibitem{first people in stereochemistry}
\hypertarget{first_people_stereochemistry}{}
\href{ https://en.wikipedia.org/wiki/Jacobus_Henricus_van_%27t_Hoff }{ Jacobus Henricus van 't Hoff }


\bibitem{quantum theory of molecula}
\hypertarget{must_molecula_have_shape}{}
\href{https://pubs.acs.org/doi/pdf/10.1021/ja00472a009} {Must a molecula have a shape ? }


\bibitem{molecular embedding}
\hypertarget{molecular_embedding}{}
\href{https://pubs.acs.org/doi/pdf/10.1021/ja00472a009} {Graph of Words Embedding for MolecularStructure-Activity Relationship Analysis }

\bibitem{handshaking lemma}
\hypertarget{handshaking_lemma}{}
\href{https://pubs.acs.org/doi/pdf/10.1021/ja00472a009} {Handshaking lemma}

\bibitem{orbit stabilizer theorem}
\hypertarget{orbit_theorem}{}
Винберг, Э. Б. Курс алгебры. — 2-е изд. — М.: Издательство «Факториал Пресс», 2014. — 179 c. — ISBN 978-5-4439-2013-9 

\bibitem{count tree on n vertex}
\hypertarget{count_tree}{}
\href{http://oeis.org/A000055} {Number of trees with n unlabeled nodes.} 

\bibitem{number of connected graphs with n nodes.}
\hypertarget{count_graphs}{}
\href{http://oeis.org/A001349} {Number of connected graphs with n nodes.} 



\end{thebibliography}

\end{document}
